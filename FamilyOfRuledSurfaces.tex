\documentclass[12pt]{amsart}
\usepackage{amsmath,amsfonts,euscript,amscd,amsthm,amssymb,upref,graphics,color}
\usepackage[all]{xy}
\usepackage[normalem]{ulem} 
%%%%%%%%%%%%%%%%%%%%%%%%%%%%%%%%%%%%%%%%%%%%%%%%%%%%%%%%%%%%%%%%%%%%

\swapnumbers

\theoremstyle{plain}

\newcommand{\red}{\color{red}}
\newcommand{\cyan}{\color{cyan}}




\newtheorem{theorem}[subsection]{Theorem}
\newtheorem*{thmnonumber}{Theorem}
\newtheorem{proposition}[subsection]{Proposition}
\newtheorem{lemma}[subsection]{Lemma}
\newtheorem{corollary}[subsection]{Corollary}
\newtheorem{conjecture}[subsection]{Conjecture}
\newtheorem*{cornonumber}{Corollary}
\newtheorem*{lemma*}{Lemma}
\newtheorem{sheafificationThm}[subsection]{Sheafification Theorem}

\newtheorem{subtheorem}{Theorem}[subsection]
\newtheorem{subproposition}[subtheorem]{Proposition}
\newtheorem{sublemma}[subtheorem]{Lemma}
\newtheorem{subcorollary}[subtheorem]{Corollary}
\newtheorem{subnothing}[subtheorem]{}

\theoremstyle{definition}

\newtheorem{definition}[subsection]{Definition}
\newtheorem{definitions}[subsection]{Definitions}
\newtheorem*{definonumber}{Definition}
\newtheorem{nothing}[subsection]{}
\newtheorem{nothing*}[subsection]{}
\newtheorem{example}[subsection]{Example}
\newtheorem{examples}[subsection]{Examples}
\newtheorem*{exnonumber}{Example}
\newtheorem*{quesnonumber}{Question}
\newtheorem{question}{Question}
\newtheorem{problem}[subsection]{Problem}	
\newtheorem{exercise}[subsection]{Exercise}	
\newtheorem*{solution}{Solution}	
\newtheorem{notation}[subsection]{Notation}
\newtheorem{notations}[subsection]{Notations}
\newtheorem{step}{Step}
\newtheorem*{claim}{Claim}
\newtheorem{assumptions}[subsection]{Assumptions}
\newtheorem{remark}[subsection]{Remark}
\newtheorem{remarks}[subsection]{Remarks}

\newtheorem{subdefinition}[subtheorem]{Definition}
\newtheorem{subnotation}[subtheorem]{Notation}
\newtheorem{subnotations}[subtheorem]{Notations}
\newtheorem{subexample}[subtheorem]{Example}
\newtheorem{subnothing*}[subtheorem]{}

\newtheorem*{warning}{Warning}
\newtheorem*{smallexample}{Example}

%%%%%%%%%%%%%%%%%%%%%%%%%%%%%%%%%%%%%%%%%%%%%%%%%%%%%%%%%%%%%%%%%%%%
\newenvironment{enumerata}%
{\begin{enumerate}
\renewcommand{\theenumi}{\alph{enumi}}
\renewcommand{\theenumii}{\roman{enumii}}}{\end{enumerate}}
%%%%%%%%%%%%%%%%%%%%%%%%%%%%%%%%%%%%%%%%%%%%%%%%%%%%%%%%%%%%%%%%%%%%


%\input macros
\DeclareMathOperator*{\Oplus}{\oplus}
\newcommand{\coker}{		\operatorname{{\rm coker}}}
\newcommand{\Aut}{		\operatorname{{\rm Aut}}}
\newcommand{\Spec}{		\operatorname{{\rm Spec}}}
\newcommand{\Proj}{		\operatorname{{\rm Proj}}}
\newcommand{\Nil}{		\operatorname{{\rm Nil}}}
\newcommand{\Der}{		\operatorname{{\rm Der}}}
\newcommand{\rank}{		\operatorname{{\rm rank}}}
\newcommand{\Reg}{		\operatorname{{\rm Reg}}}
\newcommand{\height}{		\operatorname{{\rm ht}}}
\newcommand{\supp}{		\operatorname{{\rm supp}}}
\newcommand{\image}{	\operatorname{{\rm im}}}
\newcommand{\ord}{		\operatorname{{\rm ord}}}
\newcommand{\bideg}{	\operatorname{{\rm bideg}}}
\newcommand{\trdeg}{	\operatorname{{\rm trdeg}}}
\newcommand{\Frac}{		\operatorname{{\rm Frac}}}
\newcommand{\Char}{		\operatorname{{\rm char}}}
\newcommand{\Span}{		\operatorname{{\rm Span}}}
\newcommand{\Sing}{		\operatorname{{\rm Sing}}}
\newcommand{\Pic}{		\operatorname{{\rm Pic}}}
\newcommand{\Cl}{		\operatorname{{\rm Cl}}}
\newcommand{\dom}{		\operatorname{{\rm dom}}}
\newcommand{\codom}{		\operatorname{{\rm codom}}}
\renewcommand{\div}{	\operatorname{{\rm div}}}
\newcommand{\id}{		\operatorname{{\rm id}}}
\newcommand{\ob}{		\operatorname{{\rm ob}}}
\newcommand{\lcm}{		\operatorname{{\rm lcm}}}
\newcommand{\type}{		\operatorname{{\rm type}}}
\newcommand{\Hom}{		\operatorname{{\rm Hom}}}
\newcommand{\Image}{		\operatorname{{\rm Im}}}


\newcommand{\Set}{		\operatorname{{\rm\bf Set}}}
\newcommand{\PtSet}{		\operatorname{{\rm\bf PtSet}}}
\newcommand{\Top}{		\operatorname{{\rm\bf Top}}}
\newcommand{\Grp}{		\operatorname{{\rm\bf Grp}}}
\newcommand{\CRng}{		\operatorname{{\rm\bf CRng}}}
\newcommand{\Gr}{		\operatorname{{\rm\bf Gr}}}

\newcommand{\lb}{\langle}
\newcommand{\rb}{\rangle}
\newcommand{\gr}{\operatorname{{\rm\bf gr}}}


\newcommand{\Mod}[1]{\mbox{\rm ${#1}$-\bf Mod}}
\newcommand{\setspec}[2]{\big\{\,#1\, \mid \,#2\, \big\}}

\newcommand{\powerset}{\raisebox{\depth}{\Large $\wp$}}

\newlength{\mylength}
\settowidth{\mylength}{$\,$}
\setlength{\mylength}{1.45\mylength}
\newcommand{\notdiv}{\not\hspace{\mylength}\mid}

\newcommand{\epi}{\twoheadrightarrow}
\newcommand{\overepi}[1]{\overset{ #1 }{\epi}}
\newcommand{\monic}{\rightarrowtail}
\newcommand{\overmonic}[1]{\overset{ #1 }{\monic}}

\newcommand{\Integ}{\ensuremath{\mathbb{Z}}}
\newcommand{\Nat}{\ensuremath{\mathbb{N}}}
\newcommand{\Rat}{\ensuremath{\mathbb{Q}}}
\newcommand{\Comp}{\ensuremath{\mathbb{C}}}
\newcommand{\Reals}{\ensuremath{\mathbb{R}}}
\newcommand{\aff}{\ensuremath{\mathbb{A}}}
\newcommand{\proj}{\ensuremath{\mathbb{P}}}
\newcommand{\bk}{{\ensuremath{\rm \bf k}}}
\newcommand{\ck}{{\bar{\bk}}}
\newcommand{\kk}[1]{\bk^{[#1]}}

\newcommand{\Aeul}{\EuScript{A}}
\newcommand{\Beul}{\EuScript{B}}
\newcommand{\Ceul}{\EuScript{C}}
\newcommand{\Deul}{\EuScript{D}}
\newcommand{\Eeul}{\EuScript{E}}
\newcommand{\Feul}{\EuScript{F}}
\newcommand{\Geul}{\EuScript{G}}
\newcommand{\Heul}{\EuScript{H}}
\newcommand{\Keul}{\EuScript{K}}
\newcommand{\Oeul}{\EuScript{O}}
\newcommand{\Peul}{\EuScript{P}}
\newcommand{\Seul}{\EuScript{S}}

\newcommand{\Acal}{\mathcal{A}}
\newcommand{\Bcal}{\mathcal{B}}
\newcommand{\Ccal}{\mathcal{C}}
\newcommand{\Dcal}{\mathcal{D}}
\newcommand{\Ecal}{\mathcal{E}}
\newcommand{\Fcal}{\mathcal{F}}
\newcommand{\Gcal}{\mathcal{G}}

\newcommand{\ggoth}{\mathfrak{g}}
\newcommand{\pgoth}{\mathfrak{p}}
\newcommand{\Pgoth}{\mathfrak{P}}
\newcommand{\qgoth}{\mathfrak{q}}
\newcommand{\Qgoth}{\mathfrak{Q}}
\newcommand{\mgoth}{\mathfrak{m}}
\newcommand{\Mgoth}{\mathfrak{M}}

\newcommand{\Abf}{\mathbf{A}}
\newcommand{\Bbf}{\mathbf{B}}
\newcommand{\Cbf}{\mathbf{C}}
\newcommand{\Dbf}{\mathbf{D}}
\newcommand{\Ebf}{\mathbf{E}}

\newcommand{\TX}{\mathbf{T}_{\!X}}
\newcommand{\PP}{\mbox{\boldmath $\Peul$}}
\newcommand{\pp}{\mbox{\scriptsize\boldmath $\Peul$}}
\newcommand{\SH}{\mbox{\boldmath $\Seul$}}
\newcommand{\sh}{\mbox{\scriptsize\boldmath $\Seul$}}
\newcommand{\PPP}{\mathbb{P}}
\newcommand{\SSS}{\mathbb{S}}
%\newcommand{\AAA}{\mathbb{A}}
%\newcommand{\BBB}{\mathbb{B}}
\newcommand{\AAA}{\Cbf}
\newcommand{\BBB}{\Abf}

\newcommand{\dirlim}{\varinjlim}
\newcommand{\lnd}{\operatorname{{\rm LND}}}
\newcommand{\hlnd}{\operatorname{{\rm HLND}}}
\newcommand{\klnd}{\operatorname{{\rm KLND}}}


\newcommand{\ssi}{\Leftrightarrow}
\newcommand{\isom}{\cong}
\renewcommand{\epsilon}{\varepsilon}
\renewcommand{\phi}{\varphi}
\renewcommand{\emptyset}{\varnothing}
\newcommand{\OSheaf}{\operatorname{\mathcal O}}




\newcommand{\rien}[1]{}

\newenvironment{Enumerate}[1]%
{\begin{enumerate}\setlength{\itemsep}{#1}}{\end{enumerate}}

\newenvironment{Itemize}[1]%
{\begin{itemize}\setlength{\itemsep}{#1}}{\end{itemize}}



\addtolength{\textheight}{30mm}
\setlength{\textwidth}{17.5cm}
\addtolength{\topmargin}{-15mm}
\addtolength{\oddsidemargin}{-2.5cm}
\addtolength{\evensidemargin}{-2.5cm}

\raggedbottom

\CompileMatrices

\begin{document}
\renewcommand{\baselinestretch}{1.07}



%%%%%%	TOPMATTER:   %%%%%%%%%%%%%%%%%%%%%%%%%

\title{A family of ruled surfaces and an application to Pham-Brieskorn Varieties}


\author{Michael Chitayat and Daniel Daigle}

\maketitle
  
\vfuzz=2pt


\section{Preliminaries}

Throughout this work, all rings are commutative and have a multiplicative identity which we denote by $1$. All ring homomorphisms map 1 to 1. If $B$ is a ring, then we denote its group of units by $B^*$. If $B$ is an integral domain, its fraction field will be denoted $\Frac B$. If $b \in B$, then $\lb b \rb \lhd B$ is the ideal of $B$ generated by $B$.  

A subring $A \subset B$ is \textit{factorially closed in B} if for all $x,y \in B$, we have the implication $xy \in A \Rightarrow x,y \in A$. 

A derivation $D : B \to B$ is \textit{locally nilpotent} if for all $b \in D$ there exists an $n \in \Nat$ such that $D^n(b) = 0$.  The set of locally nilpotent derivations is denoted $\lnd(B)$. The set of homogeneous locally nilpotent derivations is denoted $\hlnd(B)$.  

A derivation $D$ is \textit{irreducible} if $D(B) \subset \lb b \rb \Rightarrow b \in B^*$.    

\begin{subsection}\  	
	Let $B$ be an integral domain of characteristic zero, let $D : B \to B$ be a derivation, and let $A = \ker D$. The following facts are well-known. 
	
	(1) If $D$ is locally nilpotent, then $A$ is a factorially closed subring of $B$. 
	
	(2) If $D$ is locally nilpotent and nonzero, then $\trdeg_A(B) = 1$ and $\Frac B  = (\Frac A)^{(1)}$. 
	
	(3) Let $S \subset B$ be a multiplicative subset of $B$ containing $1$. The map $S^{-1}D : S^{-1} B \to S^{-1} B$ defined by $D(\frac{b}{s}) = \frac{sD(b)-bD(s)}{s^2}$ is a derivation satisfying:
	
	\quad (a) the derivation $S^{-1}D$ is locally nilpotent if and only if $D$ is locally nilpotent and $S \subset A$
	
	\quad (b) if $S \subset A$, then $\ker S^{-1}D = S^{-1}A$ and $S^{-1}A \cap B = A$
\end{subsection}

\section{Derivations of G-graded rings}

\begin{definition} Let B be an integral domain and let $G$ be a totally ordered abelian group. A function $\deg : B \to G \cup \{-\infty\}$  is a degree function if:

$(i) \deg(b) = -\infty \iff b = 0$

$(ii) \deg(fg) = \deg f + \deg g$ for each nonzero $f,g \in B$

$(iii) \deg(f + g) \leq \max\{\deg f, \deg g\}$ for all $f,g \in B$
\end{definition}

\begin{definition}
Let G be an abelian group, and let B be a ring. A \textit{$G$-grading of $B$} is a family $\{B_g\}_{g \in G}$ of subgroups of $B$ such that:

$(i) B = \bigoplus\limits_{g \in G} B_g$

$(ii) B_g B_h \subset B_{g+h}$ for all $g, h \in G$

\end{definition}

Suppose $B = \bigoplus\limits_{i \in G} B_i$ is a $G$-graded ring. Let $A$ be a subring of $B$.  

\begin{definitions} 
	
\begin{itemize}
	
	\item An element $b \in B_g \setminus\{0\}$ is \textit{homogeneous of degree $g$}. The element 0 is homogeneous of degree $-\infty$. 
	
	\item A derivation $D \in \Der(B)$ is  \textit{homogeneous of degree $h \in G$} if $D(B_g) \subset B_{g+h}$ for all $g \in G$. The set of homogeneous locally nilpotent derivations of $B$ is denoted $\hlnd B$. 
	
	\item Given $g \in G$, we define $A_g = A \cap B_g$. We will say that $A$ is $G$-graded if $A = \bigoplus\limits_{g \in G} A_g$ is graded. Note that we will be particularly interested in the case where $A = \ker D$ for some $D \in \hlnd(B)$. 
	
	\item Define $G(A)$ to be the subgroup of $G$ generated by the set $\{g \in G : A_g \neq 0\}$. 
	
	\item In the special case where $G = \Integ$, and $B_i = 0$ for all $i < 0$, we will say that $B$ is $\Nat$-graded and write $B = \bigoplus\limits_{i \in \Nat} B_i$. 
	
	
\end{itemize}



\end{definitions}  
\begin{remarks}
	\begin{itemize}
		\item Every $b \in B$ can be written uniquely as $b = \sum \limits_{i=1}^n b_i$ where each $b_i \in B_i$ for some $i \in G$. 		
		 
	\end{itemize} 	
\end{remarks}

We include the following useful lemma. 

\begin{lemma}\label{homoLoc}
	Let $B = \bigoplus\limits_{i \in \Nat} B_i$ be an $\Nat$-graded integral domain. Let $H$ be the set of nonzero homogeneous elements of $B$. Let $S$ be any multiplicatively closed subset of $H$. Then $S^{-1}B$ is a $\Integ$-graded rings where $\deg(\frac{b}{s}) = \deg(b) - \deg(s)$. Moreover if $D \in \hlnd(B)$ has degree $d$,  and $S \subset \ker D$, then $S^{-1}D \in \hlnd(S^{-1}B)$ and also has degree $d$.  
\end{lemma}

\section{The type of an $\Nat$-graded ring}

The following is a variation of what appears in 1.3, 1.4, and 1.5 of \cite{Daigle2007}.   

Let $B = \bk^{[n]}$ be a polynomial ring in $n$-variables and let $\ggoth$ be a grading of $B$. If the graded ring $(B, \ggoth) = \bigoplus\limits_{i \in \Nat} B_i$ satisfies $B_0 = \bk$, we will call $\ggoth$ a \textit{positive grading}. 
	
\begin{definitions}	
	Let $S = (a_1, a_2, ..., a_n) \in (\Nat \setminus\{0\})^n$ be an $n$-tuple of positive integers. We define the following values:
	
	\begin{itemize}
		\item Define $d = \gcd(\{a_1,...,a_n\})$ which we will also denote by $\gcd(S)$ 
		\item Define $L = \lcm(\{a_1, ..., a_n\})$ which we will also denote by $\lcm(S)$
		\item Define $\bar{S} = (L/a_1, L/a_2,..., L/a_n)$ and denote $L/a_i$ by $\bar{a_i}$. 
		\item Define $S_j = \{a_1, ..., a_{j-1}, \hat{a_j}, a_{j+1}, ..., a_n\}$.
		
		\item Define $\Sigma_n$ to be the group of permutations acting on $S$.  
	\end{itemize}

	as well as the following terms:
	
	\begin{itemize}
		\item If $d = 1$, we will say that $S$ is \textit{normal}. Otherwise, we can define $S'$ to be the set $\{\frac{a_1}{d}, ... , \frac{a_n}{d}\}$. Then, $S'$ is normal and we will call $S'$ the \textit{normalization of $S$}.
		
		
		
		\item Define the \textit{type} of $S$ as follows:
		
		\begin{equation}
		\type(S) = 
		\begin{cases}
		|\{j \in \{0,1,2,...,n\} : \gcd(S_j) > 1\}| & \text{if S is normal}\\
		\type(S') & \text{if S is not normal}
		\end{cases}
		\end{equation}
	\end{itemize}  

	  	
\end{definitions}

\begin{remark}
	
	If $|S| = n$ then $\type : (\Nat \setminus \{0\})^n \to \{0,1,...,n\}$. 
\end{remark}
\begin{remark}
	For all $\sigma \in \Sigma$, $\type(S) = \type(\sigma S) $. 
\end{remark}

\begin{example} We calculate the type of a few sets:
	
	$\type(\{m,m,m,m\}) = 0 \  \forall m \in \Nat \setminus \{0\}$
	
	$\type(\{m,m,n,n\}) = 0 \  \forall m,n \in \Nat \setminus \{0\}$
	
	
\end{example}

The following will be useful as it characterizes the families of Pham-Brieskorn varieties that appear in the hypotheses of Theorem 7.2 (a) and (b) of \cite{daigle:hal-01691491} as Type 1 and Type 2 varieties respectively.  

\begin{proposition}\label{equivalentType}
	Suppose $S = (a_1, a_2, ..., a_n)$ is normal, $L = \lcm (S), d = \gcd(S)$.
	
	(a) The following are equivalent. 
	
	\quad $(i) \gcd(S_j) > 1$
	
	\quad $(ii)$ there exists a prime  $p_j$ and an integer $l_j$  such that $p_j^{l_j} | \bar{a_j}$ and $p_j^{l_j} \not{|} \bar{a_i}$ for all $i \neq j$.
	
	(b) If $i \neq j$ and $\gcd(S_i)>1$,$\gcd(S_j)>1$, then $p_i$ and $p_j$ as in part (a) are distinct.    
\end{proposition}

\begin{proof}
	
	
	Write $L = p_1^{\alpha_1}p_2^{\alpha_2}...p_s^{\alpha_s}$. Let $d_j = \gcd(S_j)$ and let $p_j$ be some prime factor of $d_j$. Note that $p_j | L$. Then, 
	$$d_j > 1  \iff  p_j | a_i \text{ for all $i \neq j$ and $p_j \not{|} a_j$} \iff \bar{a_j} = L / a_j \text{ satisfies  $p_j^{\alpha_j} | \bar{a_j}$ and $p_j^{\alpha_j} \not{|} \bar{a_i}$ for any $i \neq j$}.$$
	
	This proves (a). For (b), suppose $\gcd(S_i) > 1$ and $\gcd(S_j) > 1$. By (a), we have the following two statements:
	
	$(*) \quad p_i^{l_i} | \bar{a_i}$, $p_i^{l_i} \not{|}\bar{a_j}$,
	
	$(**) \quad p_j^{l_j} | \bar{a_j}$, $p_j^{l_j} \not{|}\bar{a_i}$. 
	
	If $p_j = p_i$ then $(*)$ implies $l_i > l_j$ and $(**)$ implies that $l_j > l_i$. This is impossible so $p_i \neq p_j$. 
	
	
\end{proof}

\begin{corollary}\label{equivalentType2}
	Suppose $S = (a_1, a_2, ..., a_n)$ is normal, let $\sigma \in \Sigma_n$. Let $p_1, p_2, ..., p_k$ be primes, $\alpha_1, ..., \alpha_k \in \Nat$. Consider the following condition
	\\
	
	$(\dagger) \quad p_j^{\alpha_j} |  \overline{\sigma a_j}$ for all $j = 1,2,...,k$ and $p_j^{\alpha_j} \not{|} \overline{\sigma a_{j'}}$ for all $j' \neq j$
	\\
	 
	Then the following are equivalent. 
	
	(i) $\type(S) = k$
	
	(ii) The following two conditions hold:
	
	$\bullet$ For some $\sigma \in \Sigma_n$, there exist $k$ distinct primes $p_1,p_2,...p_k$, integers $p_1^{\alpha_1}, p_2^{\alpha_2},..., p_k^{\alpha_k}$, satisfying $(\dagger)$. 
	
	$\bullet$ For every $\sigma \in \Sigma_n$, there exist at most $k$ distinct primes $p_1,p_2,...p_k$, and integers $p_1^{\alpha_1}, p_2^{\alpha_2},..., p_k^{\alpha_k}$, satisfying $(\dagger)$.   
\end{corollary}

\begin{proof}
	This follows immediately from Proposition \ref{equivalentType}. 
\end{proof}

\begin{example} We compute the type of $S= (4,9,6,6)$ in two different ways. First, observe that $\gcd(S_1) = 3, \gcd(S_2) = 2, \gcd(S_3) = \gcd(S_4) = 1$. This shows that $\type(S) = 2$ according to the definition. 
	
	To use the method in Corollary \ref{equivalentType2} (ii), observe that $L = 36$ and hence that $\bar{S} = (9,4,6,6) = (3^2, 2^2, 2 \cdot 3, 2 \cdot 3)$. Setting $\sigma = Id, p_1 = 3, \alpha_1 = 2, i_1 = 1$ and $ p_2 = 2, \alpha_2 = 2, i_2 = 2$ satisfies the criterion in Corollary \ref{equivalentType2} (ii) and we cannot find $\sigma \in \Sigma$ and three distinct prime-integer pairs $(q_1, \alpha_1), (q_2, \alpha_2), (q_3,\alpha_3)$ to satisfy $(\dagger)$. So again, $\type(S) = 2$, by Corollary \ref{equivalentType2}.
	
	Note that $\type(6,6,9,4) = 2$ as well, however, in this case we choose $\sigma = (13)(24)$ and $p_1,\alpha_2,p_2,\alpha_2$ as above. 
	
\end{example}  

\begin{nothing} Let $B = \bk^{[n]} = \bk[X_1,X_2, ..., X_n]$ and suppose $B = \bigoplus\limits_{i \in \Nat} B_i$ is endowed with an $\Nat$-grading $\ggoth$ satisfying the following two conditions: 
\begin{itemize}
	\item $B_0 = \bk$
	\item $\deg(X_i) = d_i \geq 1$ for all $i$.
\end{itemize}

Then $\ggoth$ is a positive grading of $B = \bigoplus_{i \in \Nat} B_i$. Let $I$ be any homogeneous ideal of $B$ with respect to the grading $\ggoth$. Then $\bar{B} = B/I = \bk[x_1, ..., x_n]$ where $x_i = X_i + I$ is the image of $X_i$ under the canonical quotient homomorphism, and $\bar{B}$ is a graded ring satisfying $\deg(x_i) = d_i$ for each $i = 1, ..., n$. 
\end{nothing}

Consider now Corollary 4.2 of \cite{daigle:hal-01691491}. 

\begin{proposition} Let $B$ be a $G$-graded $\Rat$-domain, where $G$ is an abelian group. Let $n \geq 2$ and suppose that $x_1$,...,$x_n$ are homogeneous prime elements of $B$ satisfying:

$(i)$ for each subset $I \subset \{1,...,n\}$ of cardinality $n - 1$,
$\{ \deg(x_i) | i \in I\}$ generates $G(B)$.  

$(ii)$ for any choice of distinct $i,j$, $x_i, x_j$ are not associates.

Then $G(\ker D) = G(B)$ for all $G$-homogeneous $D  \in \lnd(B)$.

\end{proposition}

\section{A result on Type 0 Graded Rings}

We begin with a few definitions and well known results. 

\begin{definition}
	A field extension $L / \bk$ is \textit{ruled} if there exists a field $K$ such that $\bk \leq K \leq L$ and $L = K^{(1)}$.  
\end{definition}

\begin{notation} Let $X$ be an integral scheme. We will denote the \textit{function field of $X$} by $\Keul(X)$.
\end{notation}

\begin{lemma}\label{ffIrrelevant} Let $X = \Proj(B)$. Then $\Keul(X) = \Keul(\Spec B_{(f)})$ for any $f \in B_+$. 
\end{lemma}
\begin{proof}
	This is Proposition 2.5 in Chapter 2 of \cite{Hartshorne}, combined with the fact that the every open affine contains the unique generic point. 
\end{proof}

\begin{lemma}\label{functionFraction} Let $B$ be an integral domain. Then $\Keul(\Spec B) = \Frac B$. 
\end{lemma}

We now come to the main result of this section. 

\begin{proposition} \label{mainResult} Let $\bk$ be a field of characteristic zero, $B$ an $\Nat$-graded $\bk$-domain and let $D \in \hlnd B$. Let $A = \ker D$ and suppose that $G(A) = G(B)$. Then the function field of $X = \Proj(B)$ is ruled.
\end{proposition}

\begin{proof}
	First assume that $B$ is trivially graded. Then $\Proj B = \Spec B$ and $\Keul(X) = \Keul(\Spec B) = \Frac B = (\Frac A)^{(1)}$. So we may assume that the grading of $B$ is not trivial, and hence that $G(A) = G(B) \neq 0$.  
	
	Let $d_D = \deg D$. Since $D$ is nonzero and homogeneous, we may choose a homogeneous local slice $t \in B$. Let $u = D(t)$ and note that $u$ is homogeneous. Let $d_t = \deg(t)$. Let $g = \gcd(d_D,d_t)$. Observe that $d_t,d_D,g \in G(B)$. Now, $G(A) = G(B)$ and so there exist homogeneous $a_1,a_2 \in A$ and natural numbers $m_1,m_2$ such that $m_1\deg(a_1) - m_2\deg(a_2) = g$. It follows that there exist natural numbers $n_1,n_2,o_1,o_2$ such that $n_1\deg(a_1) - n_2 \deg(a_2) = -d_D$ and such that $o_1\deg(a_1) - o_2 \deg(a_2) = -d_t$. 
	
	Let $a = ua_1a_2$. Assume without loss of generality that $a$ is an element of the irrelevant ideal of $B$. (If not, replace $a$ by $a' = af$ for some $f$ satisfying $\deg(f) > 0$. Such an $f$ exists since $G(A) = G(B) \neq 0$.) Let $S = \{1, a, a^2, a^3, ... \}$ and observe that $S \subset H \cap A$. By Lemma \ref{homoLoc}, $B_a$ is $\Integ$-graded and the induced derivation of the localized ring $D_a: B_a \to B_a$ is $\Integ$-homogeneous of degree $d$ and is locally nilpotent. Also note that $\frac{1}{u},\frac{1}{a_1}$ and $\frac{1}{a_2} \in B_a$.  
	
	Now observe that $\frac{a_1^{n_1}}{a_2^{n_2}} \in \ker D_a$ and so $\frac{a_1^{n_1}}{a_2^{n_2}} D_a \in \hlnd(B_a)$ and has degree 0. Let us denote the degree 0 subring of $B_a$ by $B_{(a)}$. Then $\frac{a_1^{n_1}}{a_2^{n_2}} D_a$ restricts to a locally nilpotent derivation of $B_{(a)}$. Let us denote the restriction of $\frac{a_1^{n_1}}{a_2^{n_2}} D_a$ by $D' : B_{(a)} \to  B_{(a)}$.  
	
	We show that $D': B_{(a)} \to B_{(a)}$ has a slice. Indeed $\frac{ta_1^{o_1}}{a_2^{o_2}} \in B_{(a)}$ and $$D'(\frac{ta_1^{o_1}}{a_2^{o_2}}) = \frac{a_1^{o_1}a_1^{n_1}}{a_2^{o_2}a_2^{n_2}}D_a(t) = \frac{a_1^{o_1}a_1^{n_1}u}{a_2^{o_2}a_2^{n_2}} \in B_{(a)}^*.$$ 
	
	Now, let $X = \Proj B$. Since $B_{(a)}$ has a slice, we have that $\Keul(X) = \Keul(\Spec B_{(a)}) = \Frac B_{(a)}= K^{(1)}$ where $K = \Frac(\ker D')$. Note that the first equality follows from Lemma \ref{ffIrrelevant} and the fact the $a \in B_+$, the second equality from Lemma \ref{functionFraction} and the third from the fact that $D'$ has a slice. This completes the proof.     
\end{proof}

The above theorem is particularly useful in answering questions about the rigidity of Pham-Brieskorn varieties, a topic we discuss in the following section. 

\begin{corollary}
	
\end{corollary}

\section{Some results on semi-rigid rings}

\begin{definition}
	A ring $B$ is semi-rigid if $ML(B) = \ker D$ for any $D \in \lnd(B) \setminus\{0\}$. 
\end{definition}

The following results are required. They appear as Corollary 1.26, Lemmas 2.9, 2.10, 2.23 and Theorem 2.24 in \cite{freudenburg2017algebraic}.

\begin{proposition}\label{sliceTheorem}
	Suppose $D \in \lnd(B)$ has a slice $t$, and let $A = \ker D$. Then,
	
	(a) $B = A[t]$
	
	(b) $A = \pi_s(B)$ and $\ker \pi_s = sB$
	
	(c) If $B$ is affine, then $A$ is affine. 
	
\end{proposition}

\begin{lemma} \label{FV2-2.9}
	Suppose $\bk$ is algebraically closed and $B$ is an affine UFD over $\bk$ with $\trdeg_\bk B = 1$. Then $B \isom_\bk \bk[T]_{f(T)}$ for some $f(T) \in \bk[T]$. If in addition $B^* = \bk^*$, then $B = \bk^{[1]}$. 
\end{lemma}


\begin{lemma}\label{FV2-2.10}
	Suppose $\bk$ is an algebraically closed field. If $B$ is a UFD over $\bk$ with $\trdeg_\bk B = 2$, then every irreducible element of $\lnd(B)$ has a slice. 
\end{lemma}

\begin{lemma} \label{rigidLocalization}
	Let $B$ be an affine $\bk$-domain. 
	\\
	
	(a) If $B$ is rigid, then every localization of $B$ is rigid. 
	
	(b) If $B$ is semi-rigid, then every localization of $B$ is semi-rigid. 
\end{lemma}

\begin{theorem} \label{semirigidityTheorem}
	If $A$ is a rigid commutative $\bk$-domain of finite transcendence degree over $\bk$, then $A^{[1]}$ is semi-rigid.
\end{theorem}

We also require the following, originally proved by Zariski. {\red A reference for this can be found as Theorem 1.1 in "A condition for finite generation of the kernel of a derivation" by Kuroda, in the Journal of Algebra 262 (2003) with online reference:

https://www.sciencedirect.com/science/article/pii/S002186930300067X
}

\begin{lemma}\label{fgkerneldim2}
	Let $D$ be any $\bk$-derivation of a finitely generated normal domain over $\bk$, and let $A = \ker D$. If $\trdeg_\bk A = 2$, then $\ker D$ is a finitely generated $\bk$-algebra.  
\end{lemma}

The following Proposition is a special case of Proposition 3.9 in \cite{lndsStructure}. We include the proof of this special case, as the preprint of \cite{lndsStructure} has not yet been published. 

\begin{proposition}\label{trdeg3UFDnotRat}
	Suppose $\bk$ is algebraically closed of characteristic 0 and $B$ is ring with the following properties:
	
	(1) $B$ is a $\bk$-affine $UFD$
	
	(2) $B$ is not rational
	
	(3) $\trdeg_\bk B = 3$
	
	Let $A = \ker D$. Then,
	
	(a) $\ker D \not\isom \bk^{[2]}$
	
	(b) for every $A \in \klnd(B)$, $A$ is a rigid ring.
	
	(c) $B$ is semi-rigid. 
	
\end{proposition}

\begin{proof}
	If $B$ is rigid, the only locally nilpotent derivation is $D = 0$, so $\ker D = B$ is not $\bk^{[2]}$, is rigid, and hence is semi-rigid. So, suppose $B$ is not rigid. Let $D_B \in \lnd(B)$ be nonzero and let $A = \ker D_B$. Then $A$ has transcendence degree 2 over $\bk$, and since $B$ is a UFD, $A$ is also a UFD. 
	
	Note first that $A$ is not $\bk^{[2]}$ since in that case, $\Frac(A) = \bk^{(2)}$ and then $\Frac B = (\Frac A)^{(1)} = \bk^{(3)}$ which is impossible since $B$ is not rational. This proves (a).
	
	For (b), assume $A$ is not rigid. Since $A$ is a UFD, it follows that $A$ satisfies the ascending chain condition for principal ideals, and since $A$ is not rigid, there exists an irreducible derivation $D_A \in \lnd(A)$. (See Proposition 2.2 of \cite{freudenburg2017algebraic} for instance.) By Lemma \ref{FV2-2.10}, $D_A$ has a slice. It follows that $A = R^{[1]}$ where $R = \ker D_A$. If $R = \bk^{[1]}$ then $A = \bk^{[2]}$ which is impossible by part (a), so $A = R^{[1]}$ where $\trdeg_\bk R = 1$. Now, by Lemma \ref{fgkerneldim2}, $A$ is finitely generated as a $\bk$-algebra. Then, by Proposition \ref{sliceTheorem}, $R$ is also finitely generated as $\bk$-algebra. Then, by Lemma \ref{FV2-2.9}, it follows that $R \isom_\bk \bk[T]_{f(T)}$ and so $\Frac R = \bk(T)$, $\Frac(A) = (\Frac R)^{(1)} = \bk^{(2)}$ and $\Frac B = (\Frac A)^{(1)} = \bk^{(3)}$ which is impossible since $B$ is not rational. It follows that $A$ must be rigid.
	
	For (c), note that $B = A_\alpha^{[1]}$ for some $\alpha \in A$. By part (a) of Lemma \ref{rigidLocalization}, we have that $A_\alpha$ is rigid, and by Theorem of \ref{semirigidityTheorem}, $B = A_\alpha^{[1]}$ is semi-rigid.      		
\end{proof}



\section{Applications to Pham-Brieskorn Varieties}

\begin{definition}
Let  $B = \Comp[X_1,X_2,...,X_n]/ \lb X_1^{a_1} + ... + X_n^{a_n} \rb$ where $n \geq 3$.  Then $B$ is called a \textit{Pham-Brieskorn ring} and we will denote such a ring by $B_{a_1, ..., a_n}$.  The variety $\Spec B$ is called a \textit{Pham-Brieskorn variety}.
\end{definition}

Let $V$ be a Pham-Brieskorn $\Comp$-variety defined by the equation $X_1^{a_1} + ... + X_n^{a_n}$. It is conjectured that $V$ is rigid if and only if $a_i = 1$ for some $i$, or if $a_i = a_j = 2$ for some $i \neq j$. This result is known when $n = 3$ (or equivalently when $\dim V = 2$). See Theorem 8.2 of \cite{freudenburg2013} for one direction. The other directly follows from the fact that Danielewski hypersurfaces are not rigid.   

For the balance of this chapter we will show that the above conjecture holds for additional families of Pham-Brieskorn varieties.

\subsection{Some known results}

Let $B = B_{a_1, ..., a_n}$ be the coordinate ring of a Pham-Brieskorn variety. We have the following result, which appears as Example 2.6 in \cite{almostRigidRings}. 

\begin{theorem}\label{lowSum} If $\sum\limits_{i=1}^n \frac{1}{a_i} \leq \frac{1}{n-2}$ then $B$ is rigid. 
\end{theorem} 

We also include the following useful result which is Corollary 4.3 of \cite{freudenburg2013}.

\begin{proposition} \label{rigidDegree2}
	Suppose $R$ is a $\Integ$-graded affine $\bk$-domain, $f \in R$ is homogeneous,and $n \geq 2$ is an integer not dividing $\deg f$. Set $d = \gcd(n, \deg f)$, and assume that the rings
	$$ S = R[u]= \lb f + u^d \rb \text{ and } B = R[z]= \lb f + z^n \rb$$
	are domains. The following are equivalent.
	
	$(a) |u|_S \geq 2$
	
	$(b) B$ is rigid. 
\end{proposition}

Assume now that $n=4$ and so $B$ is 3-dimensional. Rather than write $B_{a_1,a_2,a_3,a_4}$, we will write $B_{a,b,c,d}$ or simply $(a,b,c,d)$. The following theorem appears in \cite{LNDsAbelianGroup}.    

\begin{theorem} \label{collection}
	Let $B = B_{a,b,c,d}$ and assume that $a,b,c,d \geq 2$ are such that at most one among $a,b,c,d$ equals 2. Then $B_{a,b,c,d}$ is rigid in each of the following cases:
	
	(a)  $\frac{1}{a} + \frac{1}{b} + \frac{1}{c} + \frac{1}{d} \leq \frac{1}{2}$
	
	(b) $\gcd(abc,d) = 1$
	
	(c) $(3,3,3,d)$
	
	(d) $(2,b,c,d)$, where $b,c,d \geq 3$, $b$ is even, $\gcd(b,c) \geq 3$, $\gcd(d,\lcm(b,c)) = 2$
	
	(e) $\type(a,b,c,d) \geq 2$
\end{theorem}

Part (a) is the application of Theorem \ref{lowSum} with $n=4$. Part (b) is due to {\red Insert reference for (b) }. For part (c), the $(3,3,3,3)$ case is due to Cheltsov, Park and Won and can be found as Corollary 1.9 in \cite{affineCones}; when $(3,d) = 1$ the $(3,3,3,d)$ case follows from (b); when $3|d$, it follows from the $(3,3,3,3)$ case and from Proposition \ref{rigidDegree2}. Part (d) is due to {\red Insert reference for (d)}; (e) is Theorem 7.2 (b) in \cite{LNDsAbelianGroup}.

Note that it is not too hard to generalize Theorem \ref{collection} (e) so that it applies to arbitrary Pham-Brieskorn varieties. 

\begin{corollary}
	Let $B = B_{a_1,...,a_n}$ be a Pham-Brieskorn variety such that $a_i \neq 1$ for all $i$, and at most one amongst the $a_i = 2$. If $\type(a_1,...,a_n) \geq n-2$, then $B$ is rigid.  
\end{corollary}

\begin{proof}
	The $n = 3,4$ cases are discussed above. We will proceed by induction with $n=4$ as our base case. Suppose the result holds for the $n-1$ case. Assume that $B = B_{a_1, ..., a_n}$ and that $\type(B) \geq n-2$. 
	
	Let $D \in \lnd(B)$ and assume without loss of generality that $D$ is irreducible. Applying Corollary 6.3(a) of \cite{LNDsAbelianGroup}, (since $\type(B) \geq 2$), there exist $x_i,x_j$ such that $D^2(x_i) = 0 = D^2(x_j)$. By Corollary 6.3(b) of \cite{LNDsAbelianGroup}, it follows that $D(x_i) = 0$ or $D(x_j) = 0$. Without loss of generality, assume $D(x_i) = 0$. Since $D$ is irreducible, $D$ induces a nontrivial locally nilpotent derivation $\bar{D}$ on $B / \lb x_i \rb \isom B_{a_1,...,a_{i-1},a_{i+1},...,a_n} \isom B_{a_1,...,a_m}$ where $m = n-1$ after reindexing. But $\type(B / \lb x_i \rb) \geq m-2$ and so $B / \lb x_i \rb$ is rigid by the induction hypothesis. This contradicts the fact that $\bar{D}$ is nontrivial.        
\end{proof}

Let $B = B_{a_1, ..., a_n} = \Comp[x_1, ..., x_n]$ be a Pham-Brieskorn ring where each $x_i$ is the image of $X_i$ under the obvious quotient homomorphism from $\Comp[X_1,...,X_n] \to B$. Let $L = \lcm(a_1, ..., a_n)$. Then we can endow $B$ with an $\Nat$-grading by setting $\deg(x_i) = L/a_i$ and thus view $B$ as an $\Nat$-graded ring. As such it makes sense to consider $\Proj B$ as a scheme. 

\begin{remark} It is well-known that when $n \geq 3$, $B$ is an integral domain. Since $B_+ \neq 0$ it follows that $\Proj B$ is an integral scheme.
\end{remark}

We recall a few facts about the Fermat hypersurfaces. 

\begin{definition} The Fermat hypersurface $F_d$ of degree $d$ in $\proj^{n-1} = \Proj \Comp[X_1, ..., X_n]$ is the projective variety defined by the homogeneous equation $X_1^d + X_2^d + ... + X_n^d$.  
\end{definition}

\begin{remark} The Fermat hypersurfaces $F_d$ are smooth for all $d \geq 1$ and all $n \geq 2$. 
\end{remark}
\begin{remark} \label{notRuled}
	If $d > n-1$, then $F_d$ is not ruled. In particular, when $d=n$, $F_d$ is a K3 variety, and when $d > n$, $F_d$ is of general type. This can be found in the discussion in Example 8.20.3 in Chapter 2 of \cite{Hartshorne}. 
\end{remark}	

\begin{corollary} Let $B = B_{a_1,a_2,...,a_n}$ be the Pham-Brieskorn variety with $a_1 = a_2 = ... = a_n = d$ where $d \geq n$. Then $B$ is rigid. 
\end{corollary}

\begin{proof} Observe first that $\type(B) = 0$ and so $\Integ(A) = \Integ(B)$. It follows from Proposition \ref{mainResult} that if $\Proj(B) = F_d$ is not rigid, then $F_d$ is ruled. But by Remark \ref{notRuled}, this is impossible. 
\end{proof}

We obtain the following Corollary in the 3-dimensional case. 

\begin{corollary}
	The ring $B_{a,b,c,d}$ is rigid in each of the following cases.
	\\
	
	(a) $(a,a,a,a)$ for $a \geq 3$
	
	(b) $(a,a,a,ka)$ for $a \geq 3, k \geq 1$
\end{corollary}
\begin{proof}
	Part (a) is due to Theorem \ref{mainResult} with $n = 4$. Part (b) then follows from $(a)$ and from Proposition \ref{rigidDegree2}. 
\end{proof}	
	
        






\bibliographystyle{plain}
\bibliography{bibliography}


\end{document}